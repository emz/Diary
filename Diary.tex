\documentclass[b5paper]{book}
\usepackage[T1]{fontenc}
\usepackage{color}

\begin{document}
\setlength{\parskip}{1em}
\noindent\textbf{%
	Sunday, March 30\textsuperscript{th} 2014\\
	\emph{Coming to Terms with Discontent with Life}
}\\

It was supposed to be a quick trip for a quick job, nothing more. But the past three days in İskenderun were surprisingly enlightening; what I had anticipated to be another boring visit to our companies manufacturing plant turned out to be the culmination ---albeit in a subtle way--- and unravelling of how poor application of business psychology was the driver behind most of my miseries in Haz.

Last Thursday, I was sitting in a restaurant where Serdal took me for dinner, at a nice spot looking over the Mediterranean, the weather was a bit chilly, but was balanced by the steaming heat of the beef dish we had as a main course. This was my third visit to the factory, and I think I had gotten to know Serdal enough to be able to easily start an interesting conversation and sustain it for a couple of hours. My first impression of him when we first spoke on the phone 2 years ago, and later during my first visit, had been of a vulgar and unpleasant person; he turned out to be a very nice and hospitable person, and I think I now know why he seemed so unpleasant.

I wouldn't look for answers to life's bigger questions in a casual chat, but that's where I found one. The view of the sea had reminded me of the other beautiful places I had the chance to see like Antalya, and I asked him if he had ever gone there. He said he hadn't and that he never has the time to.

``\emph{Why don't you take a vacation and go see it?}'', I asked.

``\emph{I don't take vacations. I haven't had a vacation in ten years.}'', he replied.

``\emph{How come? Don't you ever feel like you need a break? To release the tension?}'', I asked.

``\emph{No. I don't need to. It's all in your head}''. Simple words, from a simple man\ldots or not.

He continued to explain how work comes before everything, even his wife, and that he took only two or three days off for his honeymoon. Commitment to his work was the essence of the message he was sending, and I believe him. However, for me, that was not the most important thing he had said. The words which kept resonating in my head were: ``\emph{It's all in your head.}''.

At first, I tried to correct my statement, I said that it was not just about taking a break, but trying new things in life, and the sort. But really, any rebuttal attempt would be futile. The more I think about those words, the more things start making sense, like the inspiring TED talk by a Benedictine monk, which was a 15-minute reiteration of one message: ``\emph{Want to be happy? Be \textbf{grateful}}.''; and I can now extend the statement by saying: ``\emph{\ldots because it's all in your head}''.

But that's not just it. Simple things are rarely perceived as such. Take the example of the excellent book by Scott McCloud on \emph{Understanding Comics}, and how he explains that the power of comics comes from the simplification of the human face and figure, so that the reader can project his own image onto them. I think that would explain a lot about the human psyche. We make things more complicated than they really are so that they become more interesting, or more real ---to each person individually that is. I would say that it's fine when it comes to entertainment and art, because after all, it's either meant to be fake, or perceived as an abstraction or representation of life, and it wouldn't work any other way.

However, the problem is when people actively complicate real situations in real life, by projecting their own faulted reasoning, and using their lacking knowledge, and false indicators; which brings me back to the main subject of my overwhelming feeling of discontent with work.

For a long time, when it came to my day job, the main driving force had always been my `score' and how it compared to my colleagues. I wanted to be the top scorer, I wanted to win the race, to win the match. But is it really a race? Are we all running to beat the others to the same goal? And is the manager the only referee? Is there a defined time limit that you have to outrun?




\end{document}
